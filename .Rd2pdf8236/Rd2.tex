\documentclass[a4paper]{book}
\usepackage[times,inconsolata,hyper]{Rd}
\usepackage{makeidx}
\usepackage[utf8]{inputenc} % @SET ENCODING@
% \usepackage{graphicx} % @USE GRAPHICX@
\makeindex{}
\begin{document}
\chapter*{}
\begin{center}
{\textbf{\huge Package `msco'}}
\par\bigskip{\large \today}
\end{center}
\inputencoding{utf8}
\ifthenelse{\boolean{Rd@use@hyper}}{\hypersetup{pdftitle = {msco: Multi-Species Co-Occurrence Analyses}}}{}\begin{description}
\raggedright{}
\item[Type]\AsIs{Package}
\item[Title]\AsIs{Multi-Species Co-Occurrence Analyses}
\item[Version]\AsIs{0.1.0}
\item[Author]\AsIs{Vitalis K. Lagat [aut, cre]
(<}\url{https://orcid.org/0000-0002-9063-9187}\AsIs{>),
Guillaume Latombe [aut, ths]
(<}\url{https://orcid.org/0000-0002-8589-8387}\AsIs{>),
Cang Hui [aut, ths]
(<}\url{https://orcid.org/0000-0002-3660-8160}\AsIs{>)}
\item[Maintainer]\AsIs{Vitalis K. Lagat }\email{vitaliskim@gmail.com}\AsIs{}
\item[Description]\AsIs{Co-occurrence patterns are typically explored between species pairs. However, doing
so inevitably ignore the possible higher-order interference and interactions among three or
more species in a community. To address this, the multi-species co-occurrence (msco) index,
which calculates the joint occupancy, was recently proposed (<DOI: to be added>). By reckoning
with possible higher-order species interactions, the msco index can encapsulate nine possible
archetypes of species co-occurrence patterns and avert the statistical type II error by
examining patterns that would otherwise not be detected by any pairwise co-occurrence metrics.
To dissect the effects of neutral encounter versus trait-based processes on multi-species
co-occurrence and related higher-order interactions, generalised B-spline model has also been
proposed (<DOI: to be added>). The R package ''msco'' implements the multi-species
co-occurrence index and includes a list of functions for calculating this index for different
orders, as well as testing and visualizing species co-occurrence archetypes. This package
also implements the generalized B-spline model and related analyses.}
\item[Depends]\AsIs{R (>= 3.5.0)}
\item[License]\AsIs{GPL-3}
\item[Imports]\AsIs{scales, car, ggplot2, dplyr, magick, minpack.lm, cowplot, gtools, tibble, tidyr, caret,
pagenum, ape, import, taxize, glm2, MASS, splines2, phylogram, phytools}
\item[Encoding]\AsIs{UTF-8}
\item[LazyData]\AsIs{true}
\item[RoxygenNote]\AsIs{7.1.2}
\end{description}
\Rdcontents{\R{} topics documented:}
\inputencoding{utf8}
\HeaderA{Arch\_schem}{A schematic figure of the archetypes}{Arch.Rul.schem}
%
\begin{Description}\relax
A schematic diagram illustrating nine possible archetypes (from the null model test) of the
patterns of species co-occurrences in ecological communities. The archetypes are denoted
\eqn{\{Ai: i \in (1:9)\}}{}. See the details below.
\end{Description}
%
\begin{Usage}
\begin{verbatim}
Arch_schem()
\end{verbatim}
\end{Usage}
%
\begin{Value}
The \code{Arch\_schem} function returns a \strong{schematic diagram} of the archetypes of species
co-occurrence patterns (denoted by \eqn{\{Ai: i \in (1:9)\}}{}), with the following components:
\begin{ldescription}
\item[\code{\strong{Archetype}}] \strong{Description/Interpretation}
\item[\code{A1}] The joint occupancy value of the observed community matrix (observed; dark solid line) is above the null
model. This means the null hypothesis (i.e. a statement that imply any change in the observed patterns do
not reflect any community assembly process as underlying cause) should be rejected, confirming the presence of a
mechanism of interest being tested (Lagat \emph{et al.,} 2021a). It is typical of a community whose
species are positively associated (or aggregated) more often than would be expected by chance.
Such patterns of community structure may arise from a number of ecological mechanisms
including environmental filtering or shared habitat requirements (Cordero and Jackson,
2019).
\item[\code{A2}] The observed is greater than null expectation for \code{i = 2} but within null expectation
for \eqn{i \ge 3}{}. This implies a pairwise metric detects a non-random pattern of the community
structure, but when higher order species are considered, a random pattern is produced. This is
typical of a community whose species are aggregated more often than by chance in sites with few
species than in sites with many species (Lagat \emph{et al.,} 2021a).
\item[\code{A3}] The observed is greater than null expectation for lower orders, within null expectation for
medium orders, and less than null expectation for higher orders. This means species co-occur more
often than by chance in sites with few species, but are segregated more often than by chance in
sites with many species, depicting a community structured by two different community assembly
processes (Lagat \emph{et al.,} 2021a).
\item[\code{A4}] The observed is within null expectation for \code{i = 2} but greater than null expectation for
\eqn{i \ge 3}{}. This means when use pairwise co-occurrence is used, the null hypothesis is not
rejected, but when joint occupancy is used, the same null hypothesis is rejected. I.e., pairwise
co-occurrence fails at detecting patterns of aggregation for sites with many species, i.e. a
type II error or false negative (Lagat \emph{et al.,} 2021a).
\item[\code{A5}] The observed is within the null expectation for all orders \eqn{i \ge 2}{}, implying the
test is not statistically significant. This has been ecologically inferred to mean ecological
communities are random and that no community assembly processes or mechanisms influence their
structure (Lagat \emph{et al.,} 2021a; Cordero and Jackson, 2019; Gotelli and Sounding, 2001).
\item[\code{A6}] The observed is within the null expectation for \code{i = 2} but less than the null expectation
for \eqn{i \ge 3}{}. This means when use pairwise co-occurrence is used, the null hypothesis is not
rejected, but when joint occupancy is used, the same null hypothesis is rejected. I.e., pairwise
co-occurrence fails at detecting patterns of segregation for sites with many species, i.e. a
type II error or false negative (Lagat \emph{et al.,} 2021a).
\item[\code{A7}] The observed is less than null expectation for lower orders, within null expectation
for medium orders, and greater than null expectation for higher orders. Implying species are
segregated more often than would be expected by chance in sites with few species, but co-occur
more often than by chance in sites with many species, depicting a community structured by
two different community assembly processes (Lagat \emph{et al.,} 2021a).
\item[\code{A8}] The observed is less than null expectation for \code{i = 2} but within null expectation for
\eqn{i \ge 3}{}. This means a pairwise metric detects a non-random pattern of the community structure,
but when higher order species are considered, a random pattern is produced. This is typical of a
community whose species are segregated more often than by chance in sites with few species than
in sites with many species (Lagat \emph{et al.,} 2021a).
\item[\code{A9}] The joint occupancy value of the observed community matrix (dark solid line) is below the
null model. This means the null hypothesis should be rejected, confirming the presence of a
mechanism of interest being tested (Lagat \emph{et al.,} 2021a). It is typical of a community
structured by inter-specific competition or limiting similarity, though predation might also
generate similar patterns (Hein et al. 2014).
\end{ldescription}
\end{Value}
%
\begin{Note}\relax
\code{Arch\_schem} is not a generic function which can take in any dataset and give the outputs,
but a path to a schematic diagram saved in this package. A representational figure from empirical,
simulated or any known \code{.csv} binary data matrices can be accessed with \LinkA{Jo.plots}{Jo.plots} function.
\end{Note}
%
\begin{References}\relax
\begin{enumerate}


\item{} Cordero, R.D. and Jackson, D.A. (2019). Species-pair associations, null models, and tests of
mechanisms structuring ecological communities. \emph{Ecosphere} \strong{10.} \url{https://doi.org/10.1002/ecs2.2797}

\item{} Gotelli, N. J. and Sounding, E. (2001). Research frontiers in null model analysis. \emph{Glob. Ecol.
Biogeogr.} \strong{10}, 337-343. \url{https://doi.org/10.1046/j.1466-822X.2001.00249.x}

\item{} Hein et al. (2014). Fish introductions reveal the temperature
dependence of species interactions. \emph{Proc. R. Soc. B Biol. Sci.} \strong{281}.
\url{https://doi.org/10.1098/rspb.2013.2641}

\item{} Lagat, V. K., Latombe, G. and Hui, C. (2021a). \emph{A multi-species co-occurrence index to
avoid type II errors in null model testing}. DOI: \AsIs{<To be added>}.

\end{enumerate}

\end{References}
\inputencoding{utf8}
\HeaderA{cross\_valid}{Cross validation of the generalised B-spline model}{cross.Rul.valid}
%
\begin{Description}\relax
This function implements four different cross-validation techniques to evaluate the
predictive ability of the generalised B-spline model (\emph{sensu} Lagat \emph{et al.,} 2021b).
The four different techniques implemented are:
\begin{itemize}

\item{} validation set approach;
\item{} k -fold;
\item{} Leave-one-out-cross-validation (LOOCV), and
\item{} Repeated k-fold.

\end{itemize}

\end{Description}
%
\begin{Usage}
\begin{verbatim}
cross_valid(gbsm_obj, type = "k-fold", p, k, k_fold.repeats)
\end{verbatim}
\end{Usage}
%
\begin{Arguments}
\begin{ldescription}
\item[\code{gbsm\_obj}] An object of \code{class} \code{"gbsm"} (i.e., assigned to \LinkA{gbsm}{gbsm} function).

\item[\code{type}] The type of the cross-validation approach used. It must be
\eqn{\in \{}{}"validation.set", "k-fold", "LOOCV", "repeated.k-fold" \eqn{\}}{}

\item[\code{p}] The percentage (in decimal form) of data used in training the model. The value is used if the
cross-validation approach implemented is "validation.set".

\item[\code{k}] The value of \code{k} used in both "k-fold" and "repeated.k-fold" types of
cross-validation. This value represents the number of subsets or groups that a
given sample of data is to be split into. A value of 5 or 10 is used in practice,
as it leads to an ideal bias-variance trade-off (Lagat \emph{et al.,} 2021b).

\item[\code{k\_fold.repeats}] The number of replicates used in "repeated.k-fold" type of
cross-validation.
\end{ldescription}
\end{Arguments}
%
\begin{Details}\relax
The k-fold cross-validation approach is highly recommended due to its computational
efficiency and an acceptable bias-variance trade-off, subject to the value of \code{k}
chosen to be either 5 or 10 (Lagat \emph{et al.,} 2021b). For more details on the other
cross-validation approaches, see Lagat \emph{et al.} (2021c).
\end{Details}
%
\begin{Value}
Depending on the type of cross-validation approach implemented, the \code{cross\_valid}
function returns:
\begin{itemize}

\item{} a \code{data.frame} with the following test errors (for "validation.set"):
\begin{itemize}

\item{} \code{RMSE}:   A root mean squared error;
\item{} \code{R\_squared}:   the Pearson's \eqn{r^2}{}, and
\item{} \code{MAE}:   the mean absolute error.

\end{itemize}

\item{} an \code{array} with test errors as above including the type of the
regression model used, size of the samples, number of predictors,
type of cross-validation performed, and summary of sample sizes
(for "k-fold", "LOOCV", and "repeated.k-fold").

\end{itemize}

\end{Value}
%
\begin{References}\relax
\begin{enumerate}


\item{} Fushiki, T. (2011). Estimation of prediction error by using K-fold cross-validation.
\emph{Stat. Comput.} \strong{21}, 137-146. \url{https://doi.org/10.1007/s11222-009-9153-8}

\item{} Lagat, V. K., Latombe, G. and Hui, C. (2021b). \emph{Dissecting the effects of random
encounter versus functional trait mismatching on multi-species co-occurrence and
interference with generalised B-spline modelling}. DOI: \AsIs{<To be added>}.

\item{} Lagat, V. K., Latombe, G. and Hui, C. (2021c). \emph{\code{msco}: an R software package
for null model testing of multi-species interactions and interference with
covariates}. DOI: \AsIs{<To be added>}.

\item{} Pearson, K. (1895) VII. Note on regression and inheritance in the
case of two parents. \emph{proceedings of the royal society of London,} \strong{58}:240-242.
\url{https://doi.org/10.1098/rspl.1895.0041}

\end{enumerate}

\end{References}
%
\begin{Examples}
\begin{ExampleCode}
## Not run: 

my.path <- system.file("extdata/gsmdat", package = "msco")
setwd(my.path)
s.data <- get(load("s.data.csv")) ## Species-by-site matrix
t.data <- get(load("t.data.csv")) ## Species-by-trait matrix
p.d.mat <- get(load("p.d.mat.csv")) ## Species-by-species phylogenetic distance matrix

gbsm_obj <- msco::gbsm(s.data, t.data, p.d.mat, metric= "Simpson_eqn", d.f=4,
 order.jo=3, degree=3, n=1000, b.plots=FALSE, scat.plot=FALSE,
  response.curves=FALSE, leg=1, max.vif, max.vif2, start.range=c(-0.1,0))

val.set <- msco::cross_valid(gbsm_obj, type="validation.set", p=0.8)
val.set

kfold <- msco::cross_valid(gbsm_obj, type="k-fold", k=5)
kfold

loocv <- msco::cross_valid(gbsm_obj, type="LOOCV")
loocv

repeated.kfold <- msco::cross_valid(gbsm_obj, type="repeated.k-fold", k=5, k_fold.repeats=100)
repeated.kfold


## End(Not run)
\end{ExampleCode}
\end{Examples}
\inputencoding{utf8}
\HeaderA{gbsm}{A generalised B-spline modelling for a set of neutral and trait-based variables}{gbsm}
%
\begin{Description}\relax
This function implements the generalised B-spline model (\emph{sensu} Lagat \emph{et al.,} 2021b)
for dissecting the effects of random encounter versus functional trait mismatching on
multi-species co-occurrence and interference. Generalized linear model
(\emph{sensu} Hastie and Tibshirani, 1986) with binomial variance distribution and log link
functions employed, with predictors transformed using a linear combination of B-splines
(\emph{sensu} Curry and Schoenberg, 1988).
\end{Description}
%
\begin{Usage}
\begin{verbatim}
gbsm(
  s.data,
  t.data,
  p.d.mat,
  metric = "Simpson_eqn",
  d.f = 4,
  order.jo = 3,
  degree = 3,
  n = 1000,
  b.plots = TRUE,
  gbsm.model,
  scat.plot = TRUE,
  response.curves = TRUE,
  ylabel = TRUE,
  leg = 1,
  max.vif = 20,
  max.vif2 = 10,
  start.range = c(-0.1, 0)
)
\end{verbatim}
\end{Usage}
%
\begin{Arguments}
\begin{ldescription}
\item[\code{s.data}] A species-by-site presence/absence \code{data.frame} with entries indicating
occurrence (1) and non-occurrence (0) of species in a site.

\item[\code{t.data}] A \code{data.frame} with traits as columns and species as rows. The species must be the same as in \code{s.data}.

\item[\code{p.d.mat}] A symmetric \code{matrix} with dimension names as species and entries indicating the
phylogenetic distance between any two of them (species).

\item[\code{metric}] The type of rescaling applied to the joint occupancy metric. Available options are:
\code{Simpson\_eqn} for Simpson equivalent, \code{Sorensen\_eqn} for Sorensen equivalent, \code{raw\_prop} for the
raw form of the metric rescaled by dividing by the total number of sites, N, and \code{raw} for the
raw form of the metric without rescaling.

\item[\code{d.f}] Degrees of freedom for B-splines.

\item[\code{order.jo}] Specific number of species for which the joint occupancy is computed. To implement
generalised B-spline modelling for multiple orders, see \LinkA{gbsm\_m.orders}{gbsm.Rul.m.orders} function.

\item[\code{degree}] Degree of the B-splines.

\item[\code{n}] Number of samples for which the joint occupancy is computed. These samples are non-overlapping.
I.e., sampling is done without replacement. If the total number of combinations of \code{i} species chosen
from the total species pool \code{m}, i.e. \code{choose(m,i)}, is less than this value (\code{n}), \code{choose(m,i)} is
used as the (maximum) number of samples one can set. Otherwise sampling without replacement is
performed to select just the \code{n} samples.

\item[\code{b.plots}] Boolean value indicating if B-spline basis functions (of the first predictor) should be plotted.

\item[\code{gbsm.model}] The model used if the \code{raw} form of the metric is choosen. Availbale options are \code{"quasipoisson"}
for quasipoisson GLM or \code{"nb"} for negative binomial GLM. Other metric types strictly uses binomial GLM.

\item[\code{scat.plot}] Boolean value indicating if scatter plots between joint occupancy and its predicted
values should be plotted.

\item[\code{response.curves}] A boolean value indicating if all response curves should be plotted.

\item[\code{ylabel}] Boolean value indicating if the y label should be included in the response curves. This
parameter is added to help control the appearance of plots in \LinkA{gbsm\_m.orders}{gbsm.Rul.m.orders} function.

\item[\code{leg}] Boolean value indicating if the legend of the gbsm outputs should be included in the plots. This
parameter is added to help control the appearance of plots in \LinkA{gbsm\_m.orders}{gbsm.Rul.m.orders} function.

\item[\code{max.vif}] This parameter is used to detect and avoid multi-collinearity among covariates. Its value can be varied
to have an intermediate GBSM model (based on GLM) with  certain \code{VIF} values. Any predictor variable (from the
original model) with \code{VIF} greater than this value is removed. This can be repeated until an ideal \code{VIF} of
less or equal to a desired value is achieved.

\item[\code{max.vif2}] Like \code{max.vif}, this parameter is used to detect and avoid multi-collinearity among covariates.
Its value can be varied to have a final GBSM model (based on GLM) with certain \code{VIF} values much less than \code{max.vif}.
Any predictor variable (from the intermediate model) with \code{VIF} greater than this value is removed. This
can be repeated until an ideal \code{VIF} of less or equal to a desired value is achieved.

\item[\code{start.range}] Range of starting values for glm regression.
\end{ldescription}
\end{Arguments}
%
\begin{Value}
\code{gbsm} function returns a list containing the following outputs:
\begin{ldescription}
\item[\code{\code{order.jo}}] Order of joint occupancy
\item[\code{\code{Predictors}}] Predictor variables used in GLM regression with binomial variance
distribution function and log link function.
\item[\code{\code{Responses}}] Response variables from GLM regression with binomial variance distribution
function and log link function.
\item[\code{\code{coeff}}] Coefficients of the generalized linear model used.
\item[\code{\code{glm\_obj}}] Generalized linear model used.
\item[\code{\code{j.occs}}] Rescaled observed joint occupancies. See \code{metric}above.
\item[\code{\code{bs\_pred}}] B-spline-transformed \code{Predictors}.
\item[\code{\code{var.expld}}] Amount of variation in joint occupancy explained by the \code{Predictors}. I.e.,
it is the Pearson's \strong{\eqn{r^2}{}} between the observed and predicted values of joint occupancy.
\item[\code{\code{Original.VIFs}}] Variance inflation factors from the original GBSM model (before removing covariates exceeding \code{max.vif}).
\item[\code{\code{Intermediate.VIFs}}] Variance inflation factors from the intermediate GBSM model (after removing the 1st set of covariates exceeding \code{max.vif}).
\item[\code{\code{Final.VIFs}}] Variance inflation factors from the final GBSM model (after removing the 2nd set of covariates exceeding \code{max.vif2}).
\item[\code{\code{summary}}] summary of the regression results
\end{ldescription}
\end{Value}
%
\begin{References}\relax
\begin{enumerate}


\item{} Curry, H. B., and Schoenberg, I. J. (1988). On Pólya frequency functions IV: the
fundamental spline functions and their limits. In \emph{IJ Schoenberg Selected Papers}
(pp. 347-383). Birkhäuser, Boston, MA. \url{https://doi.org/10.1007/978-1-4899-0433-1_17}

\item{} Hastie, T., and Tibshirani, R. (1986). Generalized additive models. \emph{Stat. Sci. 1}(3),
297-310. \url{https://doi.org/10.1214/ss/1177013604}

\item{} Lagat, V. K., Latombe, G. and Hui, C. (2021a). \emph{A multi-species co-occurrence
index to avoid type II errors in null model testing}. DOI: \AsIs{<To be added>}.

\item{} Lagat, V. K., Latombe, G. and Hui, C. (2021b). \emph{Dissecting the effects of random
encounter versus functional trait mismatching on multi-species co-occurrence and
interference with generalised B-spline modelling}. DOI: \AsIs{<To be added>}.

\end{enumerate}

\end{References}
%
\begin{Examples}
\begin{ExampleCode}
## Not run: 
 my.path <- system.file("extdata/gsmdat", package = "msco")
 setwd(my.path)
 s.data <- get(load("s.data.csv")) ## Species-by-site matrix
 t.data <- get(load("t.data.csv")) ## Species-by-trait matrix
 p.d.mat <- get(load("p.d.mat.csv")) ## Species-by-species phylogenetic distance matrix

 RNGkind(sample.kind = "Rejection")
 set.seed(0)
 my.gbsm <- msco::gbsm(s.data, t.data, p.d.mat, metric = "Simpson_eqn", gbsm.model,
  d.f=4, order.jo=3, degree=3, n=1000, b.plots=TRUE, scat.plot=TRUE,
   response.curves=TRUE, leg=1, max.vif = 10, max.vif2 = 3,
    start.range=c(-0.1,0))

 head(my.gbsm$bs_pred)
 head(my.gbsm$Predictors)
 head(my.gbsm$Responses)
 my.gbsm$order.jo
 my.gbsm$var.expld
 my.gbsm$Original.VIFs
 my.gbsm$Intermediate.VIFs ## Resulting covariate VIFs after removing covariates with VIF > max.vif
 my.gbsm$Final.VIFs ## Resulting covariate VIFs after removing covariates with VIF > max.vif2

 
## End(Not run)

\end{ExampleCode}
\end{Examples}
\inputencoding{utf8}
\HeaderA{gbsm.res}{Results on generalised B-spline modelling (presented in Lagat et al., 2021b)}{gbsm.res}
%
\begin{Description}\relax
This function allows the replication of the results on generalised B-spline modelling, presented
in Lagat \emph{et al}. (2021b). Executing \code{gbsm.res()} therefore gives these outputs that are saved
as \code{.RDS} files in \code{msco}. If the codes that produced these (saved) outcomes are desired, the
codes below are made available.
\end{Description}
%
\begin{Usage}
\begin{verbatim}
gbsm.res()
\end{verbatim}
\end{Usage}
%
\begin{Value}
Returns all the results presented in Lagat \emph{et al}. (2021b). To replicate
\begin{itemize}

\item{} \strong{Figs. 1}, \strong{3}, \strong{4}, \strong{5}, and \strong{Tables 1} and \strong{S1}, execute the following code:\begin{alltt}
 my.path <- system.file("extdata/gsmdat", package = "msco")
 setwd(my.path)
 s.data <- get(load("s.data.csv")) ##Species-by-site matrix
 t.data <- get(load("t.data.csv")) ##Species-by-trait matrix
 p.d.mat <- get(load("p.d.mat.csv")) ##Species-by-species phylogenetic distance matrix
 RNGkind(sample.kind = "Rejection")
 set.seed(1)
 gb.res <- msco::gbsm_m.orders(s.data,
             t.data,
             p.d.mat,
             metric = "Simpson_eqn",
             gbsm.model,
             orders = c(2:5, 8, 10, 15),
             d.f = 4,
             degree = 3,
             n = 1000,
             k = 5,
             p = 0.8,
             type = "k-fold",
             scat.plots = TRUE,
             response.curves = TRUE,
             j.occs.distrbn = TRUE,
             mp.plots = TRUE,
             max.vif = 10,
             max.vif2 = 3,
             start.range=c(-0.1,0)
           )

 gb.res$contbn_table$`order 3`  ## Table 1
 gb.res$model.validation.table  ## Table S1
 gb.res$Original.VIFs$`order 3`
 gb.res$Intermediate.VIFs$`order 3` ## Resulting covariate VIFs after removing
                                       ## covariates with VIF > max.vif
 gb.res$Final.VIFs$`order 3` ## Resulting covariate VIFs after removing
                                        ## covariates with VIF > max.vif2


\end{alltt}

\item{} \strong{Figs. S1, 2,} and \strong{S2}, execute the following codes:
\begin{itemize}

\item{} \strong{Fig. S1}:

\end{itemize}
\begin{alltt}  remotes::install_github("jinyizju/V.PhyloMaker", force = TRUE)
  library(V.PhyloMaker)
  my.path <- system.file("extdata/gsmdat", package = "msco")
  setwd(my.path)
  s.data <- get(load("s.data.csv")) ##Species-by-site matrix
  taxa <- get(load("taxa.levels.csv")) ##Species taxa
  my.phylo.plot <- msco::s.phylo(s.data,
                            p.d.mat = NULL,
                            database = "ncbi",
                            obs.taxa = FALSE,
                            taxa.levels = taxa,
                            Obs.data = FALSE,
                            phy.d.mat = FALSE,
                            phylo.plot = TRUE)

\end{alltt}

\begin{itemize}

\item{} \strong{Fig. 2}:

\end{itemize}
\begin{alltt}  my.path <- system.file("extdata/gsmdat", package = "msco")
  setwd(my.path)
  s.data <- get(load("s.data.csv")) ##Species-by-site matrix
  t.data <- get(load("t.data.csv")) ##Species-by-trait matrix
  p.d.mat <- get(load("p.d.mat.csv")) ##Species-by-species phylogenetic distance matrix
  RNGkind(sample.kind = "Rejection")
  set.seed(1)
  my.gbsm <- msco::gbsm(s.data,
                    t.data,
                    p.d.mat,
                    metric = "Simpson_eqn",
                    gbsm.model,
                    d.f = 4,
                    order.jo = 3,
                    degree = 3,
                    n = 1000,
                    b.plots = TRUE,
                    scat.plot = FALSE,
                    response.curves = FALSE,
                    leg = 1,
                    max.vif = 10,
                    max.vif2 = 3,
                    start.range=c(-0.1,0)
                  )

\end{alltt}

\begin{itemize}

\item{} \strong{Fig. 6}:

\end{itemize}
\begin{alltt}  my.path <- system.file("extdata/gsmdat", package = "msco")
  setwd(my.path)
  s.data <- get(load("s.data.csv")) ##Species-by-site matrix
  t.data <- get(load("t.data.csv")) ##Species-by-trait matrix
  p.d.mat <- get(load("p.d.mat.csv")) ##Species-by-species phylogenetic distance matrix
  RNGkind(sample.kind = "Rejection")
  set.seed(1)
  pe <- msco::pred.error.bands(s.data,
                      t.data,
                      p.d.mat,
                      metric = "Simpson_eqn",
                      gbsm.model,
                      d.f = 4,
                      simm = 10,
                      orders = c(2:5, 8, 10, 15),
                      degree = 3,
                      n = 1000,
                      start.range = c(-0.2, 0)
                    )
\end{alltt}


\end{itemize}


\strong{Caveat:} The above codes can collectively take approximately 7 minutes to execute
(with prediction uncertainty plot taking 6 minutes alone). It took 7.3895 minutes
to run (and output results) on a 64 bit system with 8 GB RAM and 3.60 GHz CPU.
\end{Value}
%
\begin{Note}\relax
The function \LinkA{gbsm.res}{gbsm.res} is not for general use. We included it in this package to help
the readers of Lagat \emph{et al}. (2021b) paper, who may want to get a deeper understanding of how the results
presented in this paper were arrived at. It also allows deeper scrutiny of Lagat \emph{et al}. (2021b)'s
findings.
\end{Note}
%
\begin{References}\relax
Lagat, V. K., Latombe, G. and Hui, C. (2021b). \emph{Dissecting the effects
of random encounter versus functional trait mismatching on multi-species
co-occurrence and interference with generalised B-spline modelling}. DOI: \AsIs{<To be added>}.
\end{References}
%
\begin{Examples}
\begin{ExampleCode}
## Not run: 

gbs.res <- msco::gbsm.res()
gbs.res$contbn_table$`order 3`
gbs.res$model.validation.table
gbs.res$Original.VIFs$`order 3`
gbs.res$Intermediate.VIFs$`order 3`
gbs.res$Final.VIFs$`order 3`


## End(Not run)
\end{ExampleCode}
\end{Examples}
\inputencoding{utf8}
\HeaderA{gbsm\_m.orders}{Predictor's contribution and model performance assessment from the results on multiple orders of joint occupancy}{gbsm.Rul.m.orders}
%
\begin{Description}\relax
This function implements the generalised B-spline model (gbsm; \emph{sensu} Lagat et al., 2021b) for dissecting the
effects of random encounter versus functional trait mismatching on multi-species co-occurrence and interference.
Unlike \LinkA{gbsm}{gbsm} that performs gbsm for a single order of species, \LinkA{gbsm\_m.orders}{gbsm.Rul.m.orders} takes
into account multiple orders of joint occupancy. In particular: for multiple joint occupancy orders, this
function computes:
\begin{itemize}

\item{} each predictor's contribution to the explained variation in joint occupancy,
\item{} the goodness-of-fit and model performance from cross-validation, and
\item{} plots the:
\begin{itemize}

\item{} response curves,
\item{} scatter plots (between the observed and predicted joint occupancy values),
\item{} histograms of the joint occupancy frequency distribution, and
\item{} model performance plots.

\end{itemize}


\end{itemize}

\end{Description}
%
\begin{Usage}
\begin{verbatim}
gbsm_m.orders(
  s.data,
  t.data,
  p.d.mat,
  metric = "Simpson_eqn",
  orders,
  d.f = 4,
  degree = 3,
  n = 1000,
  k = 5,
  p = 0.8,
  type = "k-fold",
  gbsm.model,
  scat.plots = FALSE,
  response.curves = TRUE,
  j.occs.distrbn = FALSE,
  mp.plots = FALSE,
  max.vif = 20,
  max.vif2 = 10,
  start.range = c(-0.1, 0)
)
\end{verbatim}
\end{Usage}
%
\begin{Arguments}
\begin{ldescription}
\item[\code{s.data}] A species-by-site presence/absence \code{data.frame} with entries indicating
occurrence (1) and non-occurrence (0) of species in a site.

\item[\code{t.data}] A \code{data.frame} with traits as columns and species as rows. The species must be the same as
in \code{s.data}.

\item[\code{p.d.mat}] A symmetric \code{matrix} with \code{dimnames} as species and entries indicating the
phylogenetic distance between any two of them (species).

\item[\code{metric}] As for \LinkA{gbsm}{gbsm}.

\item[\code{orders}] Specific number of species for which the joint occupancy is computed.

\item[\code{d.f}] As for \LinkA{gbsm}{gbsm}.

\item[\code{degree}] As for \LinkA{gbsm}{gbsm}.

\item[\code{n}] As for \LinkA{gbsm}{gbsm}.

\item[\code{k}] As for \LinkA{cross\_valid}{cross.Rul.valid}.

\item[\code{p}] As for \LinkA{cross\_valid}{cross.Rul.valid}.

\item[\code{type}] As for \LinkA{cross\_valid}{cross.Rul.valid}.

\item[\code{gbsm.model}] As for \LinkA{gbsm}{gbsm}.

\item[\code{scat.plots}] Boolean value indicating if scatter plots between joint occupancy and its predicted
values should be plotted.

\item[\code{response.curves}] A boolean value indicating if all response curves for all joint occupancy
orders (\code{jo.orders}) should be plotted.

\item[\code{j.occs.distrbn}] A boolean value indicating if the histograms of the frequency distribution of
observed joint occupancy should be output.

\item[\code{mp.plots}] A boolean value indicating if the model performance plots should be output.

\item[\code{max.vif}] As for \LinkA{gbsm}{gbsm}.

\item[\code{max.vif2}] As for \LinkA{gbsm}{gbsm}.

\item[\code{start.range}] As for \LinkA{gbsm}{gbsm}.
\end{ldescription}
\end{Arguments}
%
\begin{Value}
\code{gbsm\_m.orders} function returns a list containing the following outputs:
\begin{itemize}

\item{} \code{jo.orders}:   A set of joint occupancy orders.
\item{} \code{contrbn\_table}:   A \code{list} of \code{data.frame}s consisting of:
\begin{itemize}

\item{} \code{predictor}:   A column of predictors.
\item{} \code{var.expld\_M1}:   A column of goodness-of-fit (I.e., the Pearson's \eqn{r^2}{}
between the observed and predicted values of joint occupancy when all predictors
are used in the model.
\item{} \code{var.expld\_M2}:   The Pearson's \eqn{r^2}{} between the observed and the
predicted values of joint occupancy when all predictors except the predictor
whose contribution is to be determined, are used in the model.
\item{} \code{contribution}:   Each predictor's proportion of contribution in
explaining joint occupancy. This value is given by:

\code{contribution} = \eqn{\frac{var.expld_{M1} - var.expld_{M2}}{var.expld_{M1}}}{}

\end{itemize}

\item{} \code{model.validation.table}:   A \code{data.frame} with:
\begin{itemize}

\item{} \code{orders}:   Orders of joint occupancy used.
\item{} \code{Rsquared\_gf}:   Goodness-of-fit of the model. I.e., it is the
Pearson's \strong{\eqn{r^2}{}} between the observed and predicted values of
joint occupancy, for different orders.
\item{} \code{Rsquared\_cv}:   Model performance from cross-validation.

\end{itemize}

\item{} \code{metric}:   As for \LinkA{gbsm}{gbsm}.
\item{} \code{d.f}:   As for \LinkA{gbsm}{gbsm}.
\item{} \code{n}:   As for \LinkA{gbsm}{gbsm}.
\item{} \code{degree}:   As for \LinkA{gbsm}{gbsm}.
\item{} \code{jo.orders}:   Orders of joint occupancy used.
\item{} \code{Original.VIFs} As for \LinkA{gbsm}{gbsm} (different orders).
\item{} \code{Intermediate.VIFs} As for \LinkA{gbsm}{gbsm} (different orders).
\item{} \code{Final.VIFs} As for \LinkA{gbsm}{gbsm} (different orders).

\end{itemize}

\end{Value}
%
\begin{References}\relax
\begin{enumerate}

\item{} Curry, H. B., and Schoenberg, I. J. (1988). On Pólya frequency functions IV: the
fundamental spline functions and their limits. In \emph{IJ Schoenberg Selected Papers}
(pp. 347-383). Birkhäuser, Boston, MA. \url{https://doi.org/10.1007/978-1-4899-0433-1_17}

\item{} Hastie, T., and Tibshirani, R. (1986). Generalized additive models. \emph{Stat. Sci. 1}(3),
297-310. \url{https://doi.org/10.1214/ss/1177013604}

\item{} Lagat, V. K., Latombe, G. and Hui, C. (2021a). \emph{A multi-species co-occurrence
index to avoid type II errors in null model testing}. DOI: \AsIs{<To be added>}.

\item{} Lagat, V. K., Latombe, G. and Hui, C. (2021b). \emph{Dissecting the effects of random
encounter versus functional trait mismatching on multi-species co-occurrence and
interference with generalised B-spline modelling}. DOI: \AsIs{<To be added>}.

\end{enumerate}

\end{References}
%
\begin{Examples}
\begin{ExampleCode}
## Not run: 
 my.path <- system.file("extdata/gsmdat", package = "msco")
 setwd(my.path)
 s.data <- get(load("s.data.csv")) ## Species-by-site matrix
 t.data <- get(load("t.data.csv")) ## Species-by-Trait matrix
 p.d.mat <- get(load("p.d.mat.csv")) ## Species-by-species phylogenetic distance matrix


 RNGkind(sample.kind = "Rejection")
 set.seed(1)
 jp <- msco::gbsm_m.orders(s.data, t.data, p.d.mat, gbsm.model,
  metric="Simpson_eqn", orders = c(3:5, 8, 10, 15, 20), d.f=4,
   degree=3, n=1000, k=5, p=0.8, type="k-fold", scat.plots=TRUE,
    response.curves=TRUE, j.occs.distrbn=TRUE, mp.plots=TRUE,
     max.vif = 10, max.vif2 = 4, start.range=c(-0.2,0))

 jp$contbn_table[[1]]
 jp$model.validation.table
 jp$jo.orders
 jp$Original.VIFs$`order 3`
 jp$Intermediate.VIFs$`order 3`
 jp$Final.VIFs$`order 3`

 ## Close the open plots.gbsm.pdf file before running the 2nd example
 RNGkind(sample.kind = "Rejection")
 set.seed(1)
 jp2 <- msco::gbsm_m.orders(s.data, t.data, p.d.mat, gbsm.model,
  metric="Sorensen_eqn", orders = c(3:5, 8, 10, 15, 20), d.f=4,
   degree=3, n=1000, k=5, p=0.8, type="k-fold", scat.plots=TRUE,
    response.curves=TRUE, j.occs.distrbn=TRUE, mp.plots=TRUE,
     max.vif = 10, max.vif2 = 4, start.range=c(-0.2,0))

 jp2$contbn_table[[1]]
 jp2$model.validation.table
 jp2$jo.orders
 jp2$Original.VIFs$`order 3`
 jp2$Intermediate.VIFs$`order 3`
 jp2$Final.VIFs$`order 3`

## Close the open plots.gbsm.pdf file before running the 3rd example
 RNGkind(sample.kind = "Rejection")
 set.seed(1)
 jp3 <- msco::gbsm_m.orders(s.data, t.data, p.d.mat, gbsm.model,
  metric="raw_prop", orders = c(3:5, 8, 10, 15, 20), d.f=4,
   degree=3, n=1000, k=5, p=0.8, type="k-fold", scat.plots=TRUE,
    response.curves=TRUE, j.occs.distrbn=TRUE, mp.plots=TRUE,
    max.vif = 10, max.vif2 = 4, start.range=c(-0.2,0))

 jp3$contbn_table[[1]]
 jp3$model.validation.table
 jp3$jo.orders
 jp3$Original.VIFs$`order 3`
 jp3$Intermediate.VIFs$`order 3`
 jp3$Final.VIFs$`order 3`

## Close the open plots.gbsm.pdf file before running the 4th example
 RNGkind(sample.kind = "Rejection")
 set.seed(1)
 jp4 <- msco::gbsm_m.orders(s.data, t.data, p.d.mat, gbsm.model="nb",
  metric="raw", orders = c(3:5, 8, 10, 15, 20), d.f=4,
   degree=3, n=1000, k=5, p=0.8, type="k-fold", scat.plots=TRUE,
    response.curves=TRUE, j.occs.distrbn=TRUE, mp.plots=TRUE,
    max.vif = 10, max.vif2 = 4, start.range=c(-0.2,0))

 jp4$contbn_table[[1]]
 jp4$model.validation.table
 jp4$jo.orders
 jp4$Original.VIFs$`order 3`
 jp4$Intermediate.VIFs$`order 3`
 jp4$Final.VIFs$`order 3`

## Close the open plots.gbsm.pdf file before running the 5th example
 RNGkind(sample.kind = "Rejection")
 set.seed(1)
 jp5 <- msco::gbsm_m.orders(s.data, t.data, p.d.mat, gbsm.model="quasipoisson",
  metric="raw", orders = c(3:5, 8, 10, 15, 20), d.f=4,
   degree=3, n=1000, k=5, p=0.8, type="k-fold", scat.plots=TRUE,
    response.curves=TRUE, j.occs.distrbn=TRUE, mp.plots=TRUE,
    max.vif = 10, max.vif2 = 4, start.range=c(-0.2,0))

 jp5$contbn_table[[1]]
 jp5$model.validation.table
 jp5$jo.orders
 jp5$Original.VIFs$`order 3`
 jp5$Intermediate.VIFs$`order 3`
 jp5$Final.VIFs$`order 3`

 
## End(Not run)

\end{ExampleCode}
\end{Examples}
\inputencoding{utf8}
\HeaderA{j.occ}{Expected value of joint occupancy for order \eqn{i}{} and its standard deviation}{j.occ}
%
\begin{Description}\relax
This function computes joint occupancy (the average number of sites harbouring a given number of \eqn{i}{} species
simultaneously), and its standard deviation.
\end{Description}
%
\begin{Usage}
\begin{verbatim}
j.occ(s.data, order, metric = "raw")
\end{verbatim}
\end{Usage}
%
\begin{Arguments}
\begin{ldescription}
\item[\code{s.data}] A species-by-site presence/absence matrix with entries indicating
occurrence (1) and non-occurrence (0) of species in a site.

\item[\code{order}] Specific number of species for which joint occupancy and its standard deviation
is computed.

\item[\code{metric}] The type of rescaling applied to the joint occupancy metric. Available options are:
\code{Simpson\_eqn} for Simpson equivalent, \code{Sorensen\_eqn} for Sorensen equivalent, and \code{raw} for the
raw form of index without rescaling.
\end{ldescription}
\end{Arguments}
%
\begin{Value}
Returns a \code{list} with the following outputs:
\begin{ldescription}
\item[\code{jo.val}] Joint occupancy value.
\item[\code{jo.sd}] The standard deviation of \code{jo.val}.
\end{ldescription}
\end{Value}
%
\begin{References}\relax
Lagat, V. K., Latombe, G. and Hui, C. (2021a). \emph{A multi-species co-occurrence
index to avoid type II errors in null model testing}. DOI: \AsIs{<To be added>}.
\end{References}
%
\begin{Examples}
\begin{ExampleCode}
ex.data <- read.csv(system.file("extdata", "274.csv", package = "msco"))
jo <- msco::j.occ(ex.data, order = 3, metric = "raw")
jo

ex.data2 <- read.csv(system.file("extdata", "65.csv", package = "msco"))
jo2 <- msco::j.occ(ex.data2, order = 3, metric = "raw")
jo2
\end{ExampleCode}
\end{Examples}
\inputencoding{utf8}
\HeaderA{j.occs}{Expected value of joint occupancy and its standard deviation for a range of orders}{j.occs}
%
\begin{Description}\relax
This function computes joint occupancy (the average number of sites harbouring a
given number of \eqn{i}{} species simultaneously) and its standard deviation for a
range of orders (number of species).
\end{Description}
%
\begin{Usage}
\begin{verbatim}
j.occs(s.data, orders = 1:nrow(s.data), metric = "raw")
\end{verbatim}
\end{Usage}
%
\begin{Arguments}
\begin{ldescription}
\item[\code{s.data}] A species-by-site presence/absence matrix with entries indicating
occurrence (1) and non-occurrence (0) of species in a site.

\item[\code{orders}] Range number of species for which joint occupancy and its standard
deviation is computed.

\item[\code{metric}] The type of rescaling applied to the joint occupancy metric. Available options are:
\code{Simpson\_eqn} for Simpson equivalent, \code{Sorensen\_eqn} for Sorensen equivalent, and \code{raw} for the
raw form of index without rescaling.
\end{ldescription}
\end{Arguments}
%
\begin{Value}
Returns a \code{list} with the following outputs:
\begin{ldescription}
\item[\code{jo.vals}] A vector of joint occupancy values for a range number of species (in \code{orders}).
\item[\code{jo.sds}] A vector of standard deviations of \code{jo.vals}.
\end{ldescription}
\end{Value}
%
\begin{References}\relax
Lagat, V. K., Latombe, G. and Hui, C. (2021a). \emph{A multi-species co-occurrence
index to avoid type II errors in null model testing}. DOI: \AsIs{<To be added>}.
\end{References}
%
\begin{Examples}
\begin{ExampleCode}
ex.data <- read.csv(system.file("extdata", "274.csv", package = "msco"))
jos <- msco::j.occs(ex.data, orders = 1:nrow(ex.data), metric = "raw")
jos

ex.data2 <- read.csv(system.file("extdata", "65.csv", package = "msco"))
jos2 <- msco::j.occs(ex.data2, orders = 1:nrow(ex.data), metric = "raw")
jos2
\end{ExampleCode}
\end{Examples}
\inputencoding{utf8}
\HeaderA{Jo.eng}{Joint occupancy model engine}{Jo.eng}
%
\begin{Description}\relax
This function is the engine behind the null model testing of species co-occurrence patterns,
and analyses of the joint occupancy decline and the parametric forms of this decline, for
one particular community. In particular, \LinkA{Jo.eng}{Jo.eng}:
\begin{itemize}

\item{} computes the joint occupancy (i.e. the number of sites or assemblages
harbouring multiple species simultaneously);
\item{} performs a null model test using the same index;
\item{} fits the three regression models (exponential, power law and exponential-power law)
to joint occupancy decline (\emph{sensu} Lagat \emph{et al.,} 2021a) with order (number of species);
\item{} estimates the parameter values of these models;
\item{} determines the best model among the three using AIC values;
\item{} quantifies the performance of the fitted models using the Pearson's \eqn{r^2}{};
\item{} plots the joint occupancy decline regression and null models, and
\item{} ascertains the archetypes of the patterns of species co-occurrences (from null
model test) from which inferences on the type of drivers structuralising ecological
communities can be made.

\end{itemize}

\end{Description}
%
\begin{Usage}
\begin{verbatim}
Jo.eng(
  s.data,
  algo = "sim2",
  metric = "raw",
  nReps = 999,
  dig = 3,
  s.dplot = FALSE,
  All.plots = TRUE,
  Jo.coeff = TRUE,
  my.AIC = TRUE,
  my.rsq = TRUE,
  Exp_Reg = TRUE,
  P.law_Reg = TRUE,
  Exp_p.l_Reg = TRUE,
  Obs.data = FALSE,
  Sim.data = FALSE,
  Jo_val.sim = FALSE,
  C.I_Jo_val.sim = FALSE,
  Jo_val.obs = TRUE,
  Metric = TRUE,
  Algorithm = TRUE,
  S.order = TRUE,
  nmod_stats = TRUE,
  Pt_Arch_Vals = TRUE,
  Atype = TRUE,
  p.n.plot = FALSE,
  trans = FALSE,
  lab = FALSE,
  leg = FALSE,
  m.n.plot = FALSE
)
\end{verbatim}
\end{Usage}
%
\begin{Arguments}
\begin{ldescription}
\item[\code{s.data}] A species-by-site presence/absence matrix with entries indicating
occurrence (1) and non-occurrence (0) of species in a site.

\item[\code{algo}] Randomisation algorithm used for the comparison with the null model. The
possible options to choose from are: \code{sim1}, \code{sim2}, \code{sim3}, \code{sim4}, \code{sim5}, \code{sim6},
\code{sim7}, \code{sim8}, and \code{sim9}, all from Gotelli (2000). \code{sim2} is highly recommended
(see Lagat \emph{et al.,} 2021a).

\item[\code{metric}] The type of rescaling applied to the joint occupancy metric. Available options are:
\code{Simpson\_eqn} for Simpson equivalent, \code{Sorensen\_eqn} for Sorensen equivalent, and \code{raw} for the
raw form of index without rescaling.

\item[\code{nReps}] Number of simulations used in the null model test.

\item[\code{dig}] The number of decimal places of the joint occupancy values (y axis) in the plots.
The default is 3.

\item[\code{s.dplot}] A Boolean indicating whether the standard deviation
of multi-species co-occurrence index should be included in the plots of joint occupancy
decline or not.

\item[\code{All.plots}] A Boolean indicating whether joint occupancy decline  regression
and null model plots should be output.

\item[\code{Jo.coeff}] A Boolean indicating if coefficient estimates of the joint occupancy
decline regression models should be printed.

\item[\code{my.AIC}] A Boolean indicating whether Akaike Information Criterion of the joint occupancy
decline regression models should be output or not.

\item[\code{my.rsq}] A Boolean indicating whether square of correlation coefficient between the
observed and predicted values of joint occupancy should be output.

\item[\code{Exp\_Reg}] A Boolean indicating if exponential regression parametric model should be
printed.

\item[\code{P.law\_Reg}] A Boolean indicating if power law regression parametric model should be printed.

\item[\code{Exp\_p.l\_Reg}] A Boolean indicating if exponential-power law regression parametric model
should be printed.

\item[\code{Obs.data}] A Boolean indicating if observed/empirical data should be output.

\item[\code{Sim.data}] A Boolean indicating if simulated/random data produced using any of the
simulation algorithms should be output.

\item[\code{Jo\_val.sim}] A Boolean indicating if joint occupancy values of the
simulated species-by-site presence/absence matrices should be output.

\item[\code{C.I\_Jo\_val.sim}] A Boolean indicating if 95\% confidence interval of the joint occupancy values of the
simulated data should be printed. This interval is the area under the null model.

\item[\code{Jo\_val.obs}] A Boolean indicating if joint occupancy values of the
observed species-by-site presence/absence matrices should be output.

\item[\code{Metric}] A Boolean indicating if metric used should be printed.

\item[\code{Algorithm}] A Boolean indicating if simulation algorithm used should be printed.

\item[\code{S.order}] A Boolean indicating if the number of species whose joint occupancy is computed
should be printed.

\item[\code{nmod\_stats}] A Boolean indicating whether the summary
statistics for the null model test should be output.

\item[\code{Pt\_Arch\_Vals}] A Boolean indicating if character strings indicating the location of
joint occupancy value of the observed data relative to the critical
values of the 95\% closed confidence interval for every order (number of species), should
be printed.

\item[\code{Atype}] A Boolean indicating if a character string indicating the overall archetype of
joint occupancy decline should be printed.This value must be \eqn{\in \{}{}"A1", "A2", "A3", "A4",
"A5", "A6", "A7", "A8", "A9"\eqn{\}}{} or "NA". "NA" could be the combinations of two or more
of the nine expected archetypes.

\item[\code{p.n.plot}] A Boolean indicating whether null model plot produced using the
pairwise natural metric should be output.

\item[\code{trans}] A Boolean indicating if the observed and simulated values used in
\code{p.n.plot} should be transformed by raising them to (1/100). This can be done to
compare \code{p.n.plot} with \code{All.plots} at a point where the order, \code{i = 2}.

\item[\code{lab}] A Boolean indicating if the plot labels should be added to the \code{m.n.plot}. This parameter
helps to control the appearance of plots in this function.

\item[\code{leg}] A Boolean indicating if the legend should be added to the \code{m.n.plot}. This parameter
helps to control the appearance of plots in this function.

\item[\code{m.n.plot}] A Boolean indicating whether null model plot produced using joint
occupancy metrics should be output. The default is \code{FALSE}.
\end{ldescription}
\end{Arguments}
%
\begin{Value}
\code{Jo.eng} function returns a list containing the following outputs:
\begin{ldescription}
\item[\code{\code{all.plots}}] Joint occupancy decline regression and null model plots.
\item[\code{\code{jo.coeff}}] Coefficient estimates of the joint occupancy decline regression models.
\item[\code{\code{AIC}}] Akaike information criterion of the joint occupancy decline regression models.
\item[\code{\code{r2}}] Square of correlation coefficient between the observed and predicted values of
joint occupancy.
\item[\code{\code{Exp\_reg}}] Exponential regression parametric model.
\item[\code{\code{P.law\_reg}}] Power law regression parametric model.
\item[\code{\code{Exp\_p.l\_reg}}] Exponential-power law regression parametric model.
\item[\code{\code{Obs.data}}] Observed/empirical data.
\item[\code{\code{Sim.data}}] Simulated/random data produced using any of the simulation algorithms.
\item[\code{\code{jo.val.sim}}] Joint occupancy value of the simulated species-by-site
presence/absence matrices.
\item[\code{\code{C.I\_Jo\_val.sim}}] 95\% confidence interval of the joint occupancy value of the simulated data.
\item[\code{\code{jo.val.obs}}] joint occupancy value of the observed species-by-site
presence/absence matrices.
\item[\code{\code{Metric}}] Metric used. It must be "\code{j.occ}".
\item[\code{\code{Algorithm}}] Simulation algorithm used.
\item[\code{\code{nReps}}] Number of simulations performed. This value together with the joint occupancy
value of the observed data, constitutes the sampling distribution.
\item[\code{\code{s.order}}] Number of species whose joint occupancy is computed.
\item[\code{\code{Pt\_Arch\_vals}}] Character strings indicating the location of joint occupancy value
of the observed data relative to the critical values of the 95\% closed confidence interval
of the simulated data, for every order (number of species).
\item[\code{Archetype}] A character string indicating the overall archetype from \code{Pt\_Arch\_vals}.
It must be \eqn{\in \{}{}"A1", "A2", "A3", "A4", "A5", "A6", "A7", "A8", "A9"\eqn{\}}{} or
"NA". "NA" could be the combinations of two or more
of the nine expected archetypes (see \LinkA{Arch\_schem}{Arch.Rul.schem}).
\end{ldescription}
\end{Value}
%
\begin{References}\relax
\begin{enumerate}

\item{} Lagat, V. K., Latombe, G. and Hui, C. (2021a). \emph{A multi-species co-occurrence
index to avoid type II errors in null model testing}. DOI: \AsIs{<To be added>}.

\item{} Gotelli, N. J. (2000). Null model analysis of species co-occurrence patterns.
\emph{Ecology, 81(9)}, 2606-2621. \url{https://doi.org/10.1890/0012-9658(2000)081[2606:NMAOSC]2.0.CO;2}

\end{enumerate}

\end{References}
%
\begin{Examples}
\begin{ExampleCode}
ex.data <- read.csv(system.file("extdata", "274.csv", package = "msco"))
j.en <- msco::Jo.eng(ex.data, algo="sim2", metric = "raw", nReps = 999,
           dig = 3, s.dplot = FALSE, All.plots = TRUE, Jo.coeff = TRUE,
           my.AIC = TRUE, my.rsq = TRUE, Exp_Reg = TRUE, P.law_Reg = TRUE,
           Exp_p.l_Reg = TRUE, Obs.data = FALSE, Sim.data = FALSE,
           Jo_val.sim = FALSE, C.I_Jo_val.sim = FALSE, Jo_val.obs = TRUE,
           Metric = TRUE, Algorithm = TRUE, S.order = TRUE,
           nmod_stats = TRUE, Pt_Arch_Vals = TRUE, Atype = TRUE,
           p.n.plot = FALSE, trans = FALSE, lab=FALSE, leg=FALSE, m.n.plot = FALSE)
j.en

\end{ExampleCode}
\end{Examples}
\inputencoding{utf8}
\HeaderA{Jo.plots}{Joint occupancy parametric and null model plots}{Jo.plots}
%
\begin{Description}\relax
Plots the null model and joint occupancy decline with order (number of species)
and fits the decline to exponential, power law and exponential-power law
parametric models, respectively.
\end{Description}
%
\begin{Usage}
\begin{verbatim}
Jo.plots(jo_Obj)
\end{verbatim}
\end{Usage}
%
\begin{Arguments}
\begin{ldescription}
\item[\code{jo\_Obj}] A joint occupancy model object returned by the function \LinkA{Jo.eng}{Jo.eng}.
\end{ldescription}
\end{Arguments}
%
\begin{Details}\relax
This function provides a visualization of the forms of joint
occupancy decline and null model test. It offers information on:
\begin{itemize}

\item{} the outcomes of the null model test (through the appended
archetype value on (e) plot) and
\item{} the comparisons between the forms of joint occupancy decline (through
the affixed \code{AIC} and \code{rsq} values on (b), (c) and (d) plots,
respectively).

\end{itemize}

\end{Details}
%
\begin{Value}
Produces a figure consisting of the following plots:
\begin{ldescription}
\item[\code{(a)}] Joint occupancy decline.
\item[\code{(b)}] Exponential regression of the joint occupancy decline.
\item[\code{(c)}] Power law regression of the joint occupancy decline.
\item[\code{(d)}] Exponential-power law regression of the joint occupancy decline.
\item[\code{(e)}] Null model test.
\end{ldescription}
\end{Value}
%
\begin{References}\relax
Lagat, V. K., Latombe, G. and Hui, C. (2021a). \emph{A multi-species co-occurrence
index to avoid type II errors in null model testing}. DOI: \AsIs{<To be added>}.
\end{References}
%
\begin{Examples}
\begin{ExampleCode}
## Not run: 

ex.data <- read.csv(system.file("extdata", "274.csv", package = "msco"))
jo_Obj <- msco::Jo.eng(ex.data, nReps = 999, All.plots = TRUE, s.dplot = FALSE, dig = 3)
jplots <- msco::Jo.plots(jo_Obj)
jplots

ex.data2 <- read.csv(system.file("extdata", "22.csv", package = "msco"))
jo_Obj2 <- msco::Jo.eng(ex.data2, nReps = 999, All.plots = TRUE, s.dplot = FALSE, dig = 3)
jplots2 <- msco::Jo.plots(jo_Obj2)
jplots2

ex.data3 <- read.csv(system.file("extdata", "78.csv", package = "msco"))
jo_Obj3 <- msco::Jo.eng(ex.data3, nReps = 999, All.plots = TRUE, s.dplot = FALSE, dig = 3)
jplots3 <- msco::Jo.plots(jo_Obj3)
jplots3

ex.data4 <- read.csv(system.file("extdata", "65.csv", package = "msco"))
jo_Obj4 <- msco::Jo.eng(ex.data4, nReps = 999, All.plots = TRUE, s.dplot = FALSE, dig = 3)
jplots4 <- msco::Jo.plots(jo_Obj4)
jplots4


## End(Not run)
\end{ExampleCode}
\end{Examples}
\inputencoding{utf8}
\HeaderA{Jo.res}{Results on joint occupancy index (presented in Lagat et al., 2021a)}{Jo.res}
%
\begin{Description}\relax
This function allows the replication of the results presented in Lagat \emph{et al}. (2021a). Executing
\code{Jo.res()} therefore gives these outputs that are saved as \code{.RDS} files in \code{msco}. If the codes that
produced these (saved) outcomes are desired, the codes below are made available.
\end{Description}
%
\begin{Usage}
\begin{verbatim}
Jo.res()
\end{verbatim}
\end{Usage}
%
\begin{Value}
Returns all the results presented in Lagat \emph{et al}. (2021a). To replicate these results,
execute the following code:\begin{alltt}  RNGkind(sample.kind = "Rejection")
  set.seed(39)
  my.path <- system.file("extdata/myCSVs", package = "msco")
  setwd(my.path)
  my.files <- gtools::mixedsort(list.files(path = my.path, pattern = ".csv"))
  Lag.res <- msco::mJo.eng(my.files,
                   algo = "sim2",
                   metric = "raw",
                   Archetypes = FALSE,
                   AICs = FALSE,
                   Delta_AIC = FALSE,
                   datf.Delta_AIC = TRUE,
                   param_hist = TRUE,
                   params = FALSE,
                   my.r2 = FALSE,
                   my.r2.s = TRUE,
                   best.mod2 = TRUE,
                   best.mod3 = TRUE,
                   params_c.i = TRUE
                  );Lag.res

  graphics::hist(as.numeric(Lag.res$datf.Delta_AIc$Classes), labels = TRUE,
             col = c("blue", "red", "black"), breaks=seq(0, 3, 1), xaxt = "n",
             ylab = "No. of Communities", xlab = " ",
             main = "Three parametric models compared")
  h <- graphics::hist(as.numeric(Lag.res$datf.Delta_AIc$Classes), plot = FALSE)
  graphics::axis(1, at = c(0.5,1.5,2.5), labels = c("Exponential-P.law",
             "Exponential", "Power_law"),
             tick = FALSE, padj= -1.5)
  graphics::hist(as.numeric(Lag.res$datf.Delta_AIc$Exp.pl.D_AIC), col = "blue",
             ylab = "No. of communities", xlab = "Delta_AIC", labels = FALSE,
             main = "Exponential-Power law", border = "blue",
             breaks = seq(range(as.numeric(Lag.res$datf.Delta_AIc$Exp.pl.D_AIC))[1],
                          ceiling(range(as.numeric(Lag.res$datf.Delta_AIc$Exp.pl.D_AIC))[2]),1))
  graphics::hist(as.numeric(Lag.res$datf.Delta_AIc$Exp.D_AIC), col = "red",
             ylab = "No. of communities", xlab = "Delta_AIC", labels = FALSE,
             main = "Exponential", border = "red",
             breaks = seq(range(as.numeric(Lag.res$datf.Delta_AIc$Exp.D_AIC))[1],
                          ceiling(range(as.numeric(Lag.res$datf.Delta_AIc$Exp.D_AIC))[2]), 1))
  graphics::hist(as.numeric(Lag.res$datf.Delta_AIc$Pl.D_AIC), col = "black",
             ylab = "No. of communities", xlab = "Delta_AIC", labels = FALSE,
             main = "Power law", border = "black",
             breaks = seq(range(as.numeric(Lag.res$datf.Delta_AIc$Pl.D_AIC))[1],
                          ceiling(range(as.numeric(Lag.res$datf.Delta_AIc$Pl.D_AIC))[2]), 1))

\end{alltt}


\strong{Caveat:} The above code can take approximately 8 minutes to execute. It took 8.210296 minutes to run
(and output results) on a 64 bit system with 32 GB RAM and  i7-11800H \code{@} 2.30GHz processor.
\begin{itemize}

\item{} \strong{Fig. 3} can be replicated using:

\end{itemize}
\begin{alltt} grDevices::dev.new()
 msco:::nullmod_archs()

\end{alltt}

\begin{itemize}

\item{} \strong{Fig. S1} can be replicated using:

\end{itemize}
\begin{alltt}my.path <- system.file("extdata/myCSVs", package = "msco")
setwd(my.path)
my.files <- gtools::mixedsort(list.files(path = my.path, pattern = ".csv"))
grDevices::dev.new()
msco:::richness.variances(my.files)

\end{alltt}

\end{Value}
%
\begin{Note}\relax
The function \LinkA{Jo.res}{Jo.res} is not for general use. We included it in this package to help
the readers of Lagat \emph{et al}. (2021a) paper, who may want to get a deeper understanding of how the results
presented in this paper were arrived at. It also allows deeper scrutiny of Lagat \emph{et al}. (2021a)'s
findings.
\end{Note}
%
\begin{References}\relax
Lagat, V. K., Latombe, G. and Hui, C. (2021a). \emph{A multi-species co-occurrence
index to avoid type II errors in null model testing}. DOI: \AsIs{<To be added>}.
\end{References}
%
\begin{Examples}
\begin{ExampleCode}
## Not run: 

ms.res <- msco::Jo.res()
ms.res$r2.s
ms.res$best.mod2
ms.res$best.mod3
ms.res$params_c.i

## End(Not run)
\end{ExampleCode}
\end{Examples}
\inputencoding{utf8}
\HeaderA{mJo.eng}{Joint occupancy model engine for multiple communities}{mJo.eng}
%
\begin{Description}\relax
This function is the engine behind the null model testing of species co-occurrence patterns, and
analyses of the joint occupancy decline and the parametric forms of this decline, for multiple
communities. In particular:
\begin{itemize}

\item{} It performs the null model testing of species co-occurrence patterns and generates the
archetypes depicting how joint occupancy declines with the number of species (the order
of msco) based on species-by-site presence/absence \code{.csv} data matrices. From these archetypes,
inferences can be made according to the implemented null models;
\item{} Determines the robustness of the exponential, power law and exponential-power law forms of
joint occupancy decline by computing the Pearson's \eqn{r^2}{} between the joint occupancy values
of the observed data and predicted data, for all orders of species;
\item{} Gives a summary of the total number of communities (under each and for all archetypes) whose
forms of joint occupancy decline have \eqn{r^2 > 0.95}{};
\item{} Computes the AIC and Delta AIC of joint occupancy decline regression models for all communities;
\item{} Computes the total number of communities:
\begin{itemize}

\item{} with exponential as the best form of joint occupancy decline than power law and vice versa;
\item{} with either of the three regression models (exponential, power law and exponential-power law)
having the best form of the joint occupancy decline;
\end{itemize}

\item{} Estimates the parameters of:
\begin{enumerate}

\item{} \strong{exponential:} \strong{\eqn{j^{\{i\}} = a \times exp(b \times i)}{}};
\item{} \strong{power law:} \strong{\eqn{j^{\{i\}} = a \times i^b}{}}; and
\item{} \strong{exponential-power law:} \strong{\eqn{j^{\{i\}} = a \times exp(b \times i) \times i^c}{}}

\end{enumerate}
forms of joint occupancy decline, respectively, and their 95\% confidence interval.
\end{itemize}

\end{Description}
%
\begin{Usage}
\begin{verbatim}
mJo.eng(
  my.files,
  algo = "sim2",
  metric = "raw",
  nReps = 999,
  Archetypes = FALSE,
  AICs = FALSE,
  Delta_AIC = FALSE,
  datf.Delta_AIC = FALSE,
  param_hist = FALSE,
  params = FALSE,
  best.mod2 = FALSE,
  best.mod3 = FALSE,
  params_c.i = FALSE,
  my.r2 = FALSE,
  my.r2.s = FALSE
)
\end{verbatim}
\end{Usage}
%
\begin{Arguments}
\begin{ldescription}
\item[\code{my.files}] A vector containing names of species-by-site presence/absence \code{.csv} data matrices.
The data matrices should be saved in the working directory.

\item[\code{algo}] Simulation algorithm used. The possible options to choose from are: \code{sim1},
\code{sim2}, \code{sim3}, \code{sim4}, \code{sim5}, \code{sim6}, \code{sim7}, \code{sim8}, and \code{sim9}, all from
Gotelli (2000). \code{sim2} is highly recommended (see Lagat \emph{et al.,} 2021a).

\item[\code{metric}] The type of rescaling applied to the joint occupancy metric. Available options are:
\code{Simpson\_eqn} for Simpson equivalent, \code{Sorensen\_eqn} for Sorensen equivalent, and \code{raw} for the
raw form of index without rescaling.

\item[\code{nReps}] Number of simulations used in the null model test.

\item[\code{Archetypes}] A Boolean indicating if the archetypes of the patterns
of species co-occurrences in multiple communities should be included in the output.

\item[\code{AICs}] A Boolean indicating whether the akaike information criterion (AIC) and Delta AIC
of joint occupancy decline regression models for all communities should be included in the output.

\item[\code{Delta\_AIC}] A Boolean indicating whether Delta AIC (excluding AIC) should be output.

\item[\code{datf.Delta\_AIC}] A Boolean indicating whether a \code{data.frame} with \code{Classes} and \code{Param.mods} as
columns, where the former has 1, 2 and 3 values categorizing the three parametric models that has
Delta\_AIC=0 for each communities.

\item[\code{param\_hist}] A Boolean indicating whether a histogram of the number of communities where the
three parametric forms (exponential, power law and exponential-power law) of joint occupancy
decline had the lowest AIC values.

\item[\code{params}] A Boolean indicating whether parameter estimates of the joint occupancy decline
regression models should be included in the output.

\item[\code{best.mod2}] A Boolean indicating if exponential and power law regression model comparisons
should be included in the output.

\item[\code{best.mod3}] A Boolean indicating if exponential, power law and exponential-power law
regression model comparisons should be included in the output.

\item[\code{params\_c.i}] A Boolean indicating if 95\% C.I of the parameter estimates of the joint occupancy
decline regression models should be included in the output.

\item[\code{my.r2}] A Boolean indicating if the robustness of joint occupancy decline regression models
should be computed and output.

\item[\code{my.r2.s}] A Boolean indicating if the robustness summary values of joint occupancy decline
regression models should be computed and output.
\end{ldescription}
\end{Arguments}
%
\begin{Details}\relax
\code{mJo.eng} function is useful when analyzing multiple species-by-site presence/absence
data matrices at once. If one community matrix is analyzed, the outputs of the function
\LinkA{Jo.eng}{Jo.eng} should suffice.
\end{Details}
%
\begin{Value}
\code{mJo.eng} function returns a list containing the following outputs:

\$\code{Archs}

For every community, a \code{list} consisting of:

\begin{itemize}

\item{} \AsIs{\$nmod\_stats}: A data frame with the summary statistics for the null model test; and
\item{} \AsIs{\$Archetype}: Archetypes of the patterns of species co-occurrences in ecological
communities/matrices (\code{my.files}). These archetypes must be \eqn{\in \{}{}"A1",
"A2", "A3", "A4", "A5", "A6", "A7", "A8", "A9"\eqn{\}}{} or "NA". "NA" could be the
combinations of two or more of the nine expected archetypes.
\end{itemize}

\$\code{all.AICs}

A \code{list} of \code{data.frame}s containig the following components:
\begin{ldescription}
\item[\code{\code{df}}] The number of parameters in each of the three (exponential, power law and
exponential-power law) joint occupancy decline regression models.
\item[\code{\code{aic}}] The aic values for each of the three joint occupancy decline regression models.
\item[\code{\code{delta\_aic3}}] The \code{delta\_aic} values for each of the three joint occupancy decline regression
models.
\item[\code{\code{delta\_aic2}}] The \code{delta\_aic} values for exponential and power law forms of joint occupancy
decline regression models.
\end{ldescription}
\$\code{params}

A \code{data.frame} consisting of:
\begin{ldescription}
\item[\code{\code{arch}}] The archetypes of the patterns of species co-occurrences in each of the  species-by-site
presence/absence \code{.csv} data matrices.
\item[\code{\code{a.ex}}] The \code{a} parameter estimate of the exponential form of joint occupancy decline.
\item[\code{\code{b.ex}}] The \code{b} parameter estimate of the exponential form of joint occupancy decline.
\item[\code{\code{a.pl}}] The \code{a} parameter estimate of the power law form of joint occupancy decline.
\item[\code{\code{b.pl}}] The \code{b} parameter estimate of the power law form of joint occupancy decline.
\item[\code{\code{a.expl}}] The \code{a} parameter estimate of the exponential-power law form of joint occupancy decline.
\item[\code{\code{b.expl}}] The \code{b} parameter estimate of the exponential-power law form of joint occupancy decline.
\item[\code{\code{c.expl}}] The \code{c} parameter estimate of the exponential-power law form of joint occupancy decline.
\end{ldescription}
\$\code{best.mod2}

A\code{table} containig the following components:
\begin{ldescription}
\item[\code{\code{n}}] The number of ecological communities represented by species-by-site
presence/absence \code{.csv} data matrices.
\item[\code{\code{n.lwst\_aic}}] The number of communities with exponential as the best
form of joint occupancy decline than power law.
\item[\code{\code{n.delta\_aic}}] The number of communities whose exponential and power law forms of joint occupancy
decline have \code{delta\_aic = 0}, respectively. This number must be equal to \code{n.lwst\_aic}.
\item[\code{\AsIs{\%}}] The percentage of \code{n.lwst\_aic} (or \code{n.delta\_aic}) relative to the total number of
communities (\code{n}) analyzed.
\end{ldescription}
\$\code{best.mod3}

A \code{table} containig the following components:
\begin{ldescription}
\item[\code{\code{n}}] The number of ecological communities represented by species-by-site
presence/absence \code{.csv} data matrices.
\item[\code{\code{n.lwst\_aic}}] The number of communities with exponential or power law or exponential-power
law as the best form of joint occupancy decline among the three (exponential, power law and
exponential-power law) regression models.
\item[\code{\code{n.delta\_aic}}] The number of communities whose exponential, power law and exponential-power
law forms of joint occupancy decline, respectively, have \code{delta\_aic = 0}. This number must be
equal to \code{n.lwst\_aic}.
\item[\code{\AsIs{\%}}] The percentage of \code{n.lwst\_aic} (or \code{n.delta\_aic}) relative to the total number of
communities (\code{n}) analyzed.
\end{ldescription}
\$\code{params\_c.i}

A \code{data.frame} consisting of:
\begin{ldescription}
\item[\code{\code{arch}}] The archetypes of the patterns of species co-occurrences in each of the  species-by-site
presence/absence \code{.csv} data matrices.
\item[\code{\code{n}}] The number of communities under every archetype.
\item[\code{\AsIs{ex\_\%}}] The percentages of the number of communities (under every archetype) where
exponential form of joint occupancy decline fitted better than power law.
\item[\code{\code{a.ex}}] The 95\% closed confidence interval of the \code{a} parameter estimates of the exponential
form of joint occupancy decline, under every archetype.
\item[\code{\code{b.ex}}] The 95\% closed confidence interval of the \code{b} parameter estimates of the exponential
form of joint occupancy decline, under every archetype.
\item[\code{\AsIs{p.l\_\%}}] The percentages of the number of communities (under every archetype) where
power law form of joint occupancy decline fitted better than exponential.
\item[\code{\code{a.pl}}] The 95\% closed confidence interval of the \code{a} parameter estimates of the power law
form of joint occupancy decline, under every archetype.
\item[\code{\code{b.pl}}] The 95\% closed confidence interval of the \code{b} parameter estimates of the power law
form of joint occupancy decline, under every archetype.
\item[\code{\AsIs{ex.pl\_\%}}] The percentages of the number of communities (under every archetype) where exponential-power
law form of joint occupancy decline fitted better than both the exponential and power law forms.
\item[\code{\code{a.expl}}] The 95\% closed confidence interval of the \code{a} parameter estimates of the exponential-power law
form of joint occupancy decline, under every archetype.
\item[\code{\code{b.expl}}] The 95\% closed confidence interval of the \code{b} parameter estimates of the exponential-power law
form of joint occupancy decline, under every archetype.
\item[\code{\code{c.expl}}] The 95\% closed confidence interval of the \code{c} parameter estimates of the exponential-power law
form of joint occupancy decline, under every archetype.
\end{ldescription}
\$\code{r2}

A \code{list} of \code{data.frame}s containig the following components:
\begin{ldescription}
\item[\code{\code{rsq.ex}}] \eqn{r^2}{} for the exponential form of joint occupancy decline.
\item[\code{\code{rsq.pl}}] \eqn{r^2}{} for the power law form of joint occupancy decline.
\item[\code{\code{rsq.ex.pl}}] \eqn{r^2}{} for the exponential-power law form of joint occupancy decline.
\end{ldescription}
\$\code{r2.s}
\begin{itemize}

\item{} A \code{list} containig the following components:

\$\code{rsq.per.Archs}
\begin{itemize}

\item{} \code{Archs}: Archetypes of the patterns of species co-occurrences in each
of the species-by-site presence/absence .csv data matrices.
\item{} \code{n.a}: Number of communities under each archetype.
\item{} \code{rsq.ex}: Number of communities under each archetype whose exponential
forms of joint occupancy decline have \eqn{r^2 > 0.95}{}.
\item{} \code{rsq.pl}: Number of communities under each archetype whose power
law forms of joint occupancy decline have \eqn{r^2 > 0.95}{}.
\item{} \code{rsq.ex-pl}: Number of communities under each archetype whose
exponential-power law forms of joint occupancy decline have \eqn{r^2 > 0.95}{}.

\end{itemize}


\$\code{rsq.all.Communities}
\begin{itemize}

\item{} \code{n}: Number of all communities analyzed
\item{} \code{ex}: Number of communities whose exponential forms of joint occupancy
decline have \eqn{r^2 > 0.95}{}
\item{} \code{pl}: Number of communities whose power law forms of joint occupancy
decline have \eqn{r^2 > 0.95}{}
\item{} \code{ex.pl}: Number of communities whose exponential-power law forms
of joint occupancy decline have \eqn{r^2 > 0.95}{}

\end{itemize}


\end{itemize}


\$\code{m.Jo.plots}

Produces a \code{.pdf} file with multiple figures each consisting of the following plots:

\begin{ldescription}
\item[\code{(a)}] as for \LinkA{Jo.plots}{Jo.plots}
\item[\code{(b)}] as for \LinkA{Jo.plots}{Jo.plots}
\item[\code{(c)}] as for \LinkA{Jo.plots}{Jo.plots}
\item[\code{(d)}] as for \LinkA{Jo.plots}{Jo.plots}
\item[\code{(e)}] as for \LinkA{Jo.plots}{Jo.plots}
\end{ldescription}
\end{Value}
%
\begin{References}\relax
\begin{enumerate}

\item{} Lagat, V. K., Latombe, G. and Hui, C. (2021a). \emph{A multi-species co-occurrence
index to avoid type II errors in null model testing}. DOI: \AsIs{<To be added>}.

\item{} Gotelli, N. J. (2000). Null model analysis of species co-occurrence patterns.
\emph{Ecology, 81(9)}, 2606-2621. \url{https://doi.org/10.1890/0012-9658(2000)081[2606:NMAOSC]2.0.CO;2}

\item{} Pearson, K. (1895) VII. Note on regression and inheritance in the
case of two parents. \emph{proceedings of the royal society of London,} \strong{58}:240-242.
\url{https://doi.org/10.1098/rspl.1895.0041}

\item{} Petrossian, G.A., Maxfield, M (2018). An information theory approach to hypothesis testing in
criminological research. \emph{crime science,} 7(1), 2. \url{https://doi.org/10.1186/s40163-018-0077-5}

\end{enumerate}

\end{References}
%
\begin{Examples}
\begin{ExampleCode}
## Not run: 

my.path <- system.file("extdata", package = "msco")
setwd(my.path)
my.files <- gtools::mixedsort(list.files(path = my.path, pattern = "*.csv"))
my.res <- msco::mJo.eng(my.files = my.files, algo = "sim2", Archetypes = TRUE,
             metric = "raw", nReps = 999, AICs = FALSE, params = FALSE,
             best.mod2 = FALSE, best.mod3 = FALSE, params_c.i = FALSE,
             my.r2 = FALSE, my.r2.s = FALSE)
my.res$Archs$`252.csv`

my.path2 <- system.file("extdata/myCSVs", package = "msco")
setwd(my.path2)
my.files2 <- gtools::mixedsort(list.files(path = my.path2, pattern = "*.csv"))
my.res2 <- msco::mJo.eng(my.files = my.files2[250:255], algo = "sim2", Archetypes = FALSE,
              metric = "raw", nReps = 999, AICs = FALSE, params = TRUE,
              best.mod2 = FALSE, best.mod3 = FALSE, params_c.i = FALSE,
              my.r2 = FALSE, my.r2.s = FALSE)
my.res2

my.path2 <- system.file("extdata/myCSVs", package = "msco")
setwd(my.path2)
my.files2 <- gtools::mixedsort(list.files(path = my.path2, pattern = "*.csv"))
my.res3 <- msco::mJo.eng(my.files = my.files2[250:255], algo = "sim2", Archetypes = FALSE,
              metric = "raw", nReps = 999, AICs = FALSE, params = FALSE,
              best.mod2 = FALSE, best.mod3 = FALSE, params_c.i = TRUE,
              my.r2 = FALSE, my.r2.s = FALSE)
my.res3
 
## End(Not run)
\end{ExampleCode}
\end{Examples}
\inputencoding{utf8}
\HeaderA{msco.res}{Results on \code{msco} illustration (presented in Lagat et al., 2021c)}{msco.res}
%
\begin{Description}\relax
This function allows the replication of the results on \code{msco} R package illustration paper presented
in Lagat \emph{et al}. (2021c). Executing \code{msco.res()} therefore gives these outputs that are saved
as \code{.RDS} files in \code{msco}. If the codes that produced these (saved) outcomes are desired, the
codes below are made available.
\end{Description}
%
\begin{Usage}
\begin{verbatim}
msco.res()
\end{verbatim}
\end{Usage}
%
\begin{Value}
Returns all the results presented in Lagat \emph{et al}. (2021c). To replicate
\begin{itemize}

\item{} \strong{Figs. 1}, \strong{2} and \strong{Table 2}, execute the following code:\begin{alltt}
 RNGkind(sample.kind = "Rejection")
 set.seed(14)
 ex.data <- read.csv(system.file("extdata", "251.csv", package = "msco"))
 j.en <- msco::Jo.eng(ex.data,
             algo = "sim2",
             metric = "raw",
             nReps = 999,
             dig = 3,
             s.dplot = FALSE,
             All.plots = TRUE,
             Jo.coeff = TRUE,
             my.AIC = TRUE,
             my.rsq = TRUE,
             Exp_Reg = TRUE,
             P.law_Reg = TRUE,
             Exp_p.l_Reg = TRUE,
             Obs.data = FALSE,
             Sim.data = FALSE,
             Jo_val.sim = FALSE,
             lab = FALSE,
             leg = FALSE,
             C.I_Jo_val.sim = FALSE,
             Jo_val.obs = TRUE,
             Metric = TRUE,
             Algorithm = TRUE,
             S.order = TRUE,
             nmod_stats = TRUE,
             Pt_Arch_Vals = TRUE,
             Atype = TRUE,
             p.n.plot = TRUE,
             trans = FALSE,
             m.n.plot = FALSE)

 j.en$jo.coeff ## Table 1
 j.en$AIC; j.en$r2 ## Table 2
 j.en$nmod_stats ## Table 3
 grDevices::dev.new()
 j.en$all.plots

\end{alltt}

\item{} \strong{Fig. 4}, execute the following code:\begin{alltt} RNGkind(sample.kind = "Rejection")
 set.seed(14)
 grDevices::dev.new()
 msco:::nullmod_archs2()

\end{alltt}

\item{} \strong{Fig. 5}, execute the following code:\begin{alltt}
 my.path <- system.file("extdata/gsmdat", package = "msco")
 setwd(my.path)
 s.data <- get(load("s.data.csv")) #Species-by-site matrix
 t.data <- get(load("t.data.csv")) #Species-by-trait matrix
 p.d.mat <- get(load("p.d.mat.csv")) #Species-by-species phylogenetic distance matrix
 RNGkind(sample.kind = "Rejection")
 set.seed(1)
 gb.res <- msco::gbsm_m.orders(s.data,
             t.data,
             p.d.mat,
             metric = "Simpson_eqn",
             gbsm.model,
             orders = c(3:5, 8, 10, 15, 20),
             d.f = 4,
             degree = 3,
             n = 1000,
             k = 5,
             p = 0.8,
             type = "k-fold",
             scat.plots = FALSE,
             response.curves = TRUE,
             j.occs.distrbn = FALSE,
             mp.plots = FALSE,
             max.vif = 10,
             max.vif2 = 3,
             start.range=c(-0.1,0)
           )

 gb.res$Original.VIFs$`order 3`
 gb.res$Intermediate.VIFs$`order 3` ## Resulting covariate VIFs after removing
                                       ## covariates with VIF > max.vif
 gb.res$Final.VIFs$`order 3` ## Resulting covariate VIFs after removing
                                 ## covariates with VIF > max.vif2

\end{alltt}


\end{itemize}

\end{Value}
%
\begin{Note}\relax
The function \LinkA{msco.res}{msco.res} is not for general use. We included it in this package to help
the readers of Lagat \emph{et al}. (2021c) paper, who may want to get a deeper understanding of how the results
presented in this paper were arrived at. It also allows deeper scrutiny of Lagat \emph{et al}. (2021c)'s
findings, and broader understanding of the main functionalities of \code{msco} R package.
\end{Note}
%
\begin{References}\relax
\begin{enumerate}

\item{} Lagat, V. K., Latombe, G. and Hui, C. (2021a). \emph{A multi-species co-occurrence
index to avoid type II errors in null model testing}. DOI: \AsIs{<To be added>}.

\item{} Lagat, V. K., Latombe, G. and Hui, C. (2021b). \emph{Dissecting the effects of random
encounter versus functional trait mismatching on multi-species co-occurrence and
interference with generalised B-spline modelling}. DOI: \AsIs{<To be added>}.

\item{} Lagat, V. K., Latombe, G. and Hui, C. (2021c). \emph{\code{msco}: an R software package
for null model testing of multi-species interactions and interference with
covariates}. DOI: \AsIs{<To be added>}.


\end{enumerate}

\end{References}
%
\begin{Examples}
\begin{ExampleCode}
## Not run: 

ms.res <- msco::msco.res()
ms.res$nmod_stats ## Table 2


## End(Not run)
\end{ExampleCode}
\end{Examples}
\inputencoding{utf8}
\HeaderA{pred.error.bands}{Prediction uncertainty}{pred.error.bands}
%
\begin{Description}\relax
This function plots the response curves showing the effect of the predictors (i.e. trait-based and
neutral forces) on joint occupancy as the response variable, with prediction error bands (as the
standard deviation from the mean of the response variable) for all orders of joint occupancy.
\end{Description}
%
\begin{Usage}
\begin{verbatim}
pred.error.bands(
  s.data,
  t.data,
  p.d.mat,
  metric = "Simpson_eqn",
  gbsm.model,
  d.f = 4,
  simm = 10,
  orders,
  degree = 3,
  n = 1000,
  max.vif = 40,
  max.vif2 = 30,
  start.range = c(-0.1, 0)
)
\end{verbatim}
\end{Usage}
%
\begin{Arguments}
\begin{ldescription}
\item[\code{s.data}] A species-by-site presence/absence \code{data.frame} with entries indicating
occurrence (1) and non-occurrence (0) of species in a site.

\item[\code{t.data}] A \code{data.frame} with traits as columns and species as rows. The species must be the same as
in \code{s.data}.

\item[\code{p.d.mat}] A symmetric \code{matrix} with \code{dimnames} as species and entries indicating the
phylogenetic distance between any two of them (species).

\item[\code{metric}] As for \LinkA{gbsm\_m.orders}{gbsm.Rul.m.orders}.

\item[\code{gbsm.model}] As for \LinkA{gbsm\_m.orders}{gbsm.Rul.m.orders}.

\item[\code{d.f}] As for \LinkA{gbsm\_m.orders}{gbsm.Rul.m.orders}.

\item[\code{simm}] Number of Monte Carlo simulations performed

\item[\code{orders}] As for \LinkA{gbsm\_m.orders}{gbsm.Rul.m.orders}

\item[\code{degree}] As for \LinkA{gbsm\_m.orders}{gbsm.Rul.m.orders}.

\item[\code{n}] As for \LinkA{gbsm\_m.orders}{gbsm.Rul.m.orders}.

\item[\code{max.vif}] As for \LinkA{gbsm}{gbsm}.

\item[\code{max.vif2}] As for \LinkA{gbsm}{gbsm}.

\item[\code{start.range}] As for \LinkA{gbsm\_m.orders}{gbsm.Rul.m.orders}.
\end{ldescription}
\end{Arguments}
%
\begin{Value}
\code{pred.error.bands} function returns:
\begin{ldescription}
\item[\code{\code{predictors}}] a \code{data.frame} of predictors
\item[\code{\code{responses}}] a \code{data.frame} of response values of predictors
\item[\code{\code{responses.sim\_stats}}] a \code{data.frame} of the reponses' mean and standard deviation
(from \code{simm} replicates), and
\end{ldescription}
\begin{itemize}

\item{} the response curves with prediction error bands for all orders of joint occupancy

\end{itemize}

\end{Value}
%
\begin{References}\relax
\begin{enumerate}

\item{} Lagat, V. K., Latombe, G. and Hui, C. (2021a). \emph{A multi-species co-occurrence
index to avoid type II errors in null model testing}. DOI: \AsIs{<To be added>}.

\item{} Lagat, V. K., Latombe, G. and Hui, C. (2021b). \emph{Dissecting the effects of random
encounter versus functional trait mismatching on multi-species co-occurrence and
interference with generalised B-spline modelling}. DOI: \AsIs{<To be added>}.


\end{enumerate}

\end{References}
%
\begin{Examples}
\begin{ExampleCode}
## Not run: 
 my.path <- system.file("extdata/gsmdat", package = "msco")
 setwd(my.path)
 s.data <- get(load("s.data.csv")) ## Species-by-site matrix
 t.data <- get(load("t.data.csv")) ## Species-by-Trait matrix
 p.d.mat <- get(load("p.d.mat.csv")) ## Species-by-species phylogenetic distance matrix

 RNGkind(sample.kind = "Rejection")
 set.seed(1)
 pe <- msco::pred.error.bands(s.data, t.data, p.d.mat, metric="Simpson_eqn", d.f=4, simm=10,
  orders = c(2:5, 8, 10, 15), degree=3, n=1000, gbsm.model, start.range=c(-0.2, 0))

 pe$predictors$`order 2`
 pe$responses$`order 2`
 pe$responses.sim_stats$`order 2`

 pe$predictors$`order 3`
 pe$responses$`order 3`
 pe$responses.sim_stats$`order 3`

 pe$predictors$`order 10`
 pe$responses$`order 10`
 pe$responses.sim_stats$`order 10`

 
## End(Not run)

\end{ExampleCode}
\end{Examples}
\inputencoding{utf8}
\HeaderA{s.phylo}{Species phylogeny generator}{s.phylo}
%
\begin{Description}\relax
This  function generates the phylogeny of species and plots the phylogenetic tree. In
particular, given a species-by-site matrix (community), \LinkA{s.phylo}{s.phylo}:
\begin{itemize}

\item{} uses the \LinkA{tax\_name}{tax.Rul.name} function to obtain (from the
\Rhref{https://www.ncbi.nlm.nih.gov/}{NCBI} or \Rhref{https://www.itis.gov/}{ITIS} online
databases) the genus and family taxa levels of species in the community.
If NCBI is used, getting an API key is recommended. See \LinkA{tax\_name}{tax.Rul.name}
for more information. NCBI is used as default in this function;
\item{} uses the \code{phylo.maker} function to obtain the phylogeny (an object
of \code{class}: "\code{phylo}") of species in the community using taxa obtained above;
\item{} computes the phylogenetic distance matrix using the \LinkA{cophenetic.phylo}{cophenetic.phylo}
function and the phylogeny obtained above as input;
\item{} plots the phylogenetic tree using the phylogeny obtained above.

\end{itemize}

\end{Description}
%
\begin{Usage}
\begin{verbatim}
s.phylo(
  s.data,
  p.d.mat,
  database = "ncbi",
  obs.taxa = FALSE,
  taxa.levels = NULL,
  Obs.data = FALSE,
  phy.d.mat = TRUE,
  phylo.plot = TRUE
)
\end{verbatim}
\end{Usage}
%
\begin{Arguments}
\begin{ldescription}
\item[\code{s.data}] A species-by-site presence/absence \code{data.frame} with entries indicating
occurrence (1) and non-occurrence (0) of species in a site. The rows should have species'
scientific names following \Rhref{https://en.wikipedia.org/wiki/Binomial_nomenclature}{binomial nomenclature},
with no initials.

\item[\code{p.d.mat}] As for \LinkA{gbsm}{gbsm}.

\item[\code{database}] The online database used to obtain the taxonomic names (species,
genus and family) for a given rank (species list in this function). The options
are "ncbi" (default) or "itis".

\item[\code{obs.taxa}] A Boolean indicating if \code{taxa.levels} should be included in the returned list.

\item[\code{taxa.levels}] Species taxa (i.e. a \code{data.frame} with species, genus, and family as \code{colnames}) used
in extracting phylogenetic distance matrix between species. If supplied, \code{taxa.levels} won't be
computed from online repositories. Taxa provision is highly recommended.

\item[\code{Obs.data}] A Boolean indicating if \code{s.data} should be included in the returned list.

\item[\code{phy.d.mat}] A Boolean indicating if phylogenetic distance matrix should be in the returned list.

\item[\code{phylo.plot}] Boolean value indicating if the phylogenetic tree (cluster dendrogram) should be plotted.
\end{ldescription}
\end{Arguments}
%
\begin{Value}
Returns a \code{list} with the following outputs:
\begin{itemize}

\item{} \code{s.data}:  A \code{data.frame} with sites as columns and species as rows.
\item{} \code{taxa.levels}:  A \code{data.frame} with the following columns:
\begin{itemize}

\item{} \code{species}:  Species names in \code{s.data}.
\item{} \code{genus}:  Genus names of species in \code{s.data}.
\item{} \code{family}:  Family names of species in \code{s.data}.

\end{itemize}

\item{} \code{p.d.matrix}:  A symmetric \code{matrix} with dimension names as species and entries indicating the
phylogenetic distance between any two of them (species).
\item{} \code{phylo.plot}:  A phylogenetic tree (cluster dendrogram) of species in \code{s.data}

\end{itemize}

\end{Value}
%
\begin{References}\relax
\begin{enumerate}

\item{} Binomial nomenclature' (2020) \emph{Wikipedia}. Available at:
\url{https://en.wikipedia.org/wiki/Binomial_nomenclature} (Accessed: 09 November 2020).

\item{} Lagat, V. K., Latombe, G. and Hui, C., 2021b. \emph{Dissecting the effects of random
encounter versus functional trait mismatching on multi-species co-occurrence and
interference with generalised B-spline modelling}. DOI: \AsIs{<To be added>}.

\end{enumerate}

\end{References}
%
\begin{Examples}
\begin{ExampleCode}
## Not run: 

remotes::install_github("jinyizju/V.PhyloMaker", force = TRUE)
library(V.PhyloMaker)
my.path <- system.file("extdata/gsmdat", package = "msco")
setwd(my.path)
s.data <- get(load("s.data.csv"))
taxa <- get(load("taxa.levels.csv"))

my.s.phylo <- msco::s.phylo(s.data, p.d.mat = NULL, database = "ncbi", obs.taxa=TRUE,
 taxa.levels = taxa, Obs.data=TRUE, phy.d.mat=TRUE, phylo.plot = TRUE)

my.s.data <- my.s.phylo$s.data
my.s.data

my.taxa <- my.s.phylo$taxa.levels
my.taxa

my.p.d.mat <- my.s.phylo$phylogenetic.distance.matrix
my.p.d.mat


## End(Not run)
\end{ExampleCode}
\end{Examples}
\printindex{}
\end{document}
